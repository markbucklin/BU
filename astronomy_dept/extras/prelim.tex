% This file contains all the necessary setup and commands to create
% the preliminary pages according to the buthesis.sty option.

\title{LONG TITLE HERE}

\author{FIRST M LAST}

% Type of document prepared for this degree:
%   1 = Master of Science thesis,
%   2 = Doctor of Philisophy dissertation.
%   3 = Master of Science thesis and Doctor of Philisophy dissertation.
\degree=2

\prevdegrees{M.A., Boston University, Boston, MA, YEAR. \\ B.S., School, City, ST, YR}

\department{Department of Astronomy}

% Degree year is the year the diploma is expected, and defense year is
% the year the dissertation is written up and defended. Often, these
% will be the same, except for January graduation, when your defense
% will be in the fall of year X, and your graduation will be in
% January of year X+1
\defenseyear{2009}
\degreeyear{2010}

% For each reader, specify appropriate label {First, second, third},
% then name, then title. Warning: If you have more than five readers
% you are out of luck, because it will overflow to a new page.
% Sometimes you may wish to put part of the title in with the name
\reader{First}{First M. Reader, PhD}{Professor of Astronomy}
\reader{Second}{Second M. Reader, PhD}{Professor of Astronomy}
%\reader{Third}{First M. Last}{Assistant Professor of \ldots}

% The Major Professor is the same as the first reader, but must be
% specified again for the abstract page
\majorprof{First M. Reader}{\mbox{Professor of Astronomy.}}


%                       PRELIMINARY PAGES
% According to the BU guide the preliminary pages consist of:
% title, copyright (optional), approval,  acknowledgments (opt.),
% abstract, preface (opt.), Table of contents, List of tables (if
% any), List of illustrations (if any). The \tableofcontents,
% \listoffigures, and \listoftables commands can be used in the
% appropriate places. For other things like preface, do it manually
% with something like \newpage\section*{Preface}.

% This is an additional page (do not hand it in at the library) to print
% boxed-in title, author and degree statement so that they are visible through
% the opening in BU covers used for reports. This makes a nicely bound copy.
\buecethesistitleboxpage

% Make the titlepage based on the above information.  If you need
% something special and can't use the standard form, you can specify
% the exact text of the titlepage yourself.  Put it in a titlepage
% environment and leave blank lines where you want vertical space.
% The spaces will be adjusted to fill the entire page.
\maketitle

% The copyright page is blank except for the notice at the bottom. You
% must provide your name in capitals.

%\copyrightpage

% Now include the approval page based on the readers information
\approvalpage

% The acknowledgment page should go here. Use something like
% \newpage\section*{Acknowledgments} followed by your text.
\newpage
\chapter*{Acknowledgments}
Some acknowledgements here









% The abstractpage environment sets up everything on the page except
% the text itself.  The title and other header material are put at the
% top of the page, and the supervisors are listed at the bottom.  A
% new page is begun both before and after.  Of course, an abstract may
% be more than one page itself.  If you need more control over the
% format of the page, you can use the abstract environment, which puts
% the word "Abstract" at the beginning and single spaces its text.

\begin{abstractpage}
Abstract here
\end{abstractpage}

% Now you can include a preface. Again, use something like
% \newpage\section*{Preface} followed by your text

% Table of contents comes after preface
\tableofcontents

% If you have tables, uncomment the following line
\newpage
\listoftables

% If you have figures, uncomment the following line
\newpage
\listoffigures

% List of Abbrevs is NOT optional (Martha Wellman likes all abbrevs listed)
\chapter*{List of Abbreviations}
\begin{center}
  \begin{longtable}{lll}
%    \hspace*{2em} & \hspace*{1in} & \hspace*{4.5in} \\
    TLA  & ~~~~~ & Three Letter Acronym \\
    OAH  & ~~~~~ & Other Acronyms Here \\
  \end{longtable}
\end{center}


% END OF THE PRELIMINARY PAGES
%\newpage
%\phantom{.}
\vspace{4in}

\begin{singlespace}
\begin{quote}
  \textit{Facilis descensus Averni;}\\
  \textit{Noctes atque dies patet atri janua Ditis;}\\*
  \textit{Sed revocare gradum, superasque evadere ad auras,}\\
  \textit{Hoc opus, hic labor est.}\hfill{Virgil (from Don's thesis!)}
\end{quote}
\end{singlespace}

% \vspace{0.7in}
%
% \noindent
% [The descent to Avernus is easy; the gate of Pluto stands open night
% and day; but to retrace one's steps and return to the upper air, that
% is the toil, that the difficulty.]


\newpage
\endofprelim
